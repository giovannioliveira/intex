% This template is public domain.
\documentclass{ltugboat}

\usepackage{graphicx}
\usepackage{ifpdf}
\ifpdf
\usepackage[breaklinks,hidelinks]{hyperref}
\else
\usepackage{url}
\fi

\newsavebox{\intexbox}
\savebox{\intexbox}{%%
        I\hspace{-0.5pt}\textsubscript{N}\hspace{-2.5pt}\TeX}%%
\newcommand{\intex}{\usebox\intexbox}

\newcommand{\tbs}{\textbackslash}
\usepackage{scrextend}

%%% Start of metadata %%%

\title{Generating interactive documents on \intex}

% repeat info for each author.
\author{Giovanni Aparecido Oliveira}
\address{Department of Computer Science \\ Federal University of Rio de Janeiro \\ Brazil}
\netaddress{giovanniapsoliveira (at) gmail dot com}
%\personalURL{http://example.org/~user/}

%%% End of metadata %%%

\begin{document}
\maketitle

%\begin{abstract}
%TODO
%\end{abstract}
%
%\section{Introduction}
%TODO
%This is an example article for \TUB, linked from
%\url{http://tug.org/TUGboat/location.html}.
%
%We recommend the \texttt{graphicx} package for image inclusions, and the
%\texttt{hyperref} package if active urls are desired (in the \acro{PDF}
%output).  Nowadays \TUB\ is produced using \acro{PDF} files exclusively.
%
%The \texttt{ltugboat} class provides these abbreviations (and many more):
%
%% verbatim blocks are often better in \small
%\begin{verbatim}[\small]
%\AllTeX \AMS \AmS \AmSLaTeX \AmSTeX \aw \AW
%\BibTeX \CTAN \DTD \HTML
%\ISBN \ISSN \LaTeXe
%\mf \MFB
%\plain \POBox \PS
%\SGML \TANGLE \TB \TP
%\TUB \TUG \tug
%\UNIX \XeT \WEB \WEAVE
%
%\, \bull \Dash \dash \hyph
%
%\acro{FRED} -> {\small[er] fred}  % please use!
%\cs{fred}   -> \fred
%\meta{fred} -> <fred>
%\nth{n}     -> 1st, 2nd, ...
%\sfrac{3/4} -> 3/4
%\booktitle{Book of Fred}
%\end{verbatim}
%
%For references to other \TUB\ issue, please use the format
%\textsl{volno:issno}, e.g., ``\TUB\ 32:1'' for our \nth{100} issue.
%
%This file is just a template.  The \TUB\ style documentation is the
%\texttt{ltubguid} document at \url{http://ctan.org/pkg/tugboat}.  (For
%\CTAN\ references, where sensible we recommend that form of url, using
%\texttt{/pkg/}; or, if you need to refer to a specific file location,
%\texttt{http://mirror.ctan.org/\textsl{path}}.)
%
%Email \verb|tugboat@tug.org| if problems or questions.

%\section{Overview}
%\intex\space is a software toolkit to produce interactive \HTML documents. It consists in a \TeX\space package and a conversion application.\\
%The \TeX\space package defines elements to index and insert interactive content along the working document. After importing \texttt{intex.sty} in the file's preamble, the first command to be used is \texttt{\tbs newclass} which initializes a new float environment. For example, calling:
%\begin{verbatim}[\small]
%\newclass{video}{Video}{List of Videos}
%\end{verbatim}
%Initializes an environment called \texttt{video}. Entry frames can be inserted through the commands \texttt{\tbs videoframe} and \texttt{\tbs videogframe} or 
\section{\intex\space package}
\begin{labeling}{newcommand}
\item [newclass] aa
\end{labeling}


\bibliographystyle{plain}  % we recommend the plain bibliography style
\nocite{book-minimal}      % just making the bibliography non-empty
\bibliography{xampl}       % xampl.bib comes with BibTeX

\makesignature
\end{document}
