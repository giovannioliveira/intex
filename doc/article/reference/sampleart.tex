% This template is public domain.
\documentclass{ltugboat}

\usepackage{graphicx}
\usepackage{ifpdf}
\ifpdf
\usepackage[breaklinks,hidelinks]{hyperref}
\else
\usepackage{url}
\fi

%%% Start of metadata %%%

\title{Example \TUB\ article}

% repeat info for each author.
\author{First Last}
\address{Street Address \\ Town, Postal \\ Country}
\netaddress{user (at) example dot org}
\personalURL{http://example.org/~user/}

%%% End of metadata %%%

\begin{document}

\maketitle

\begin{abstract}
This is an example article for \TUB{}.
Please write an abstract.
\end{abstract}

\section{Introduction}

This is an example article for \TUB, linked from
\url{http://tug.org/TUGboat/location.html}.

We recommend the \texttt{graphicx} package for image inclusions, and the
\texttt{hyperref} package if active urls are desired (in the \acro{PDF}
output).  Nowadays \TUB\ is produced using \acro{PDF} files exclusively.

The \texttt{ltugboat} class provides these abbreviations (and many more):

% verbatim blocks are often better in \small
\begin{verbatim}[\small]
\AllTeX \AMS \AmS \AmSLaTeX \AmSTeX \aw \AW
\BibTeX \CTAN \DTD \HTML
\ISBN \ISSN \LaTeXe
\mf \MFB
\plain \POBox \PS
\SGML \TANGLE \TB \TP
\TUB \TUG \tug
\UNIX \XeT \WEB \WEAVE

\, \bull \Dash \dash \hyph

\acro{FRED} -> {\small[er] fred}  % please use!
\cs{fred}   -> \fred
\meta{fred} -> <fred>
\nth{n}     -> 1st, 2nd, ...
\sfrac{3/4} -> 3/4
\booktitle{Book of Fred}
\end{verbatim}

For references to other \TUB\ issue, please use the format
\textsl{volno:issno}, e.g., ``\TUB\ 32:1'' for our \nth{100} issue.

This file is just a template.  The \TUB\ style documentation is the
\texttt{ltubguid} document at \url{http://ctan.org/pkg/tugboat}.  (For
\CTAN\ references, where sensible we recommend that form of url, using
\texttt{/pkg/}; or, if you need to refer to a specific file location,
\texttt{http://mirror.ctan.org/\textsl{path}}.)

Email \verb|tugboat@tug.org| if problems or questions.

\bibliographystyle{plain}  % we recommend the plain bibliography style
\nocite{book-minimal}      % just making the bibliography non-empty
\bibliography{xampl}       % xampl.bib comes with BibTeX

\makesignature
\end{document}
